\documentclass[12pt,a4paper]{article}
\usepackage[utf8]{inputenc}
\usepackage[T1]{fontenc}
\usepackage[french]{babel}
\usepackage{amsmath,amssymb}
\usepackage{lmodern}
\usepackage{graphicx}
\usepackage[dvipsnames,svgnames]{xcolor}
\usepackage{hyperref}
\usepackage{lipsum}
\usepackage{xcolor}
\usepackage{marvosym}
\title{Initiation au logiciel latex}
\author{Michel Nassalang}
\date{11 mars 2020}
\begin{document}
\maketitle
\tableofcontents
\newpage
\section{Faux texte A}
%figure commentaire
{\color{blue!20!red} \bf \lipsum[5]}
\section{Faux texte B}
\lipsum[1]
\section{Faux texte C}
\textsf{\lipsum[7]}
\section{Faux texte D}
\lipsum[8]
\section{Faux texte E}
\subsection{Texte 1}
\lipsum[10]
\subsection{Texte 2}
\lipsum[11]
\section*{Faux texte F}
\lipsum[4]
\vspace{2 cm}
Un texte simple fera l'entrée en matière de notre {\color{blue!20!red} \bf cours qui se fait en classe } .

Il n'est pas \hspace{1 cm} important si\hspace{1 cm} on met un ou plusieurs espaces

$\sqrt{2}$ %vfill pour espacer formellement sur la verticale
$$\int_a^b f(x)dx $$
Modou \hfill Diagne \hfill Fada

\symbol{92}

$$ A=\{ x \in \mathbb{R} \} $$

$$ f^-1(B)=\{x \in E , f(x) \in B \} $$
\begin{itemize}
\item pain
\item lait
\item farine
\end{itemize}
\begin{enumerate}
\item [$\square $]faire ruisseler l'oignon et le beurre,
\item ajouter du citron et le riz et laisser évaporer,
\item ajouter le bouillons peu à peu ,
\item\dots
\end{enumerate}
\begin{itemize}
\item A faire:
\begin{enumerate}
\item Sortir le chien\item Rentrer les poubelles
\end{enumerate}
\item Ce qui sera fait:
\begin{enumerate}
\item Apprendre les leçons
\item Dormir
\end{enumerate}
\end{itemize}
\newpage
\begin{enumerate}
\item aaa
\begin{enumerate}
\item[a] bbb
\begin{enumerate}
\item ccc
\item ddd
\item eee
\end{enumerate}
\item bb
\end{enumerate}
\item ggg
\begin{enumerate}
\item hhh
\item  iii
\item  jjj
\item  kkk
\item  lll
\end{enumerate}
\item mmmm
\item nnn
\item AAA:ooo
\item BBB:ppp
\end{enumerate}
\begin{tabular}{|c|p{2cm}|r|}
\hline
titi toto & tutu tata  & tete\\
\hline
titi toto & toto tata  & tete\\
\hline
\end{tabular}
 

\begin{tabular}{|c|p{2cm}|r|}
\hline
 AAA & BBB & CCC\\
\hline
D & EE & FFF\\
\hline
 GGG & H & II\\
 \hline
\end{tabular}
\newpage
\begin{minipage}[t]{0.5\textwidth}
{\bfseries \small M. Jules A. B Diao}\\[.45ex]
{\small Étudiant en classe préparatoire 1}\\
{\small Institut Polytechnique de Saint Louis (IPSL)}\\
{\small Université Gaston Berger (UGB)}\\
{\small BP 234 - Saint Louis}\\[.45ex]
\Telefon- +221 78 165 02 61\\% avec \usepackage{marvosym}
\Letter- jules.diao@ugb.edu.sn\\
\end{minipage} \hspace{0.2cm}
\begin{minipage}[t]{0.5\textwidth}
\includegraphics[height=4cm,width=4cm]{king.png}
\lipsum[1]
\lipsum[2]
\end{minipage}

\end{document}