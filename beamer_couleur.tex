\documentclass{beamer}

\usepackage[french]{babel}
\usepackage[T1]{fontenc}
\usepackage[latin1]{inputenc}

\usetheme{Warsaw}

\hypersetup{pdfpagemode=FullScreen}

\colorlet{titre}{yellow}
\definecolor{vertmoyen}{RGB}{51,110,23}
\definecolor{rouge}{HTML}{DD0000}


\begin{document}
\setbeamertemplate{background canvas}{\includegraphics[width=\paperwidth,height=\paperheight]{eclat.png}} % Width pour la largeur, height pour la hauteur de l'image

% \setbeamercolor{background canvas}{bg=yellow!10!white}
%permet d'amener une couleur en background
\begin{frame}
\frametitle{\textcolor{titre}{Exemple d'utilisation d'une couleur}}
\framesubtitle{Remarquez que mon titre est en jaune}


Texte n'ayant pas encore subi de changement de couleur.\\

\color{blue}

\textcolor{vertmoyen}{Ce texte sera en vert !\\}
\textcolor{rouge}{Celui là en rouge\\}
	
Mais celui là en bleu !
	
\begin{center}
\colorbox{teal}{\textcolor[HTML]{FFFFFF}{Ah et voici un texte surligné. :D}}
\end{center}

\end{frame}

\end{document}
