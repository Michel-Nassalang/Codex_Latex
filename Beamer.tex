\documentclass[]{beamer}

%\userpackage[latin1]{inputenc}
\usepackage[utf8]{inputenc}
\usepackage[T1]{fontenc}
\usepackage[french]{babel}
\usepackage{xcolor,colortbl,array,graphicx}
\usepackage{times}
\usepackage{amsthm,amsmath,amsfonts,amssymb,amstext}
\usepackage{lipsum}
\usepackage{hyperref,url}

\title[Scolarisation UEMOA]{}
% ----------------------
\author[Michel Nassalang]{Michel~Nassalang }
% ----------------------
\institute[IPSL]{
\inst{1}%
Département d'informatique\\
Université Gaston Berger de Saint-Louis \and \inst{2}%
Département de Philosophie Théorique\\
Université d'Ailleurs}
% ----------------------
\date[]{Présentation Arduino}

% ----------------------
\mode <presentation> {
\usetheme{Montpellier}
%{Bergen}{PaloAlto}{Berkeley}{Boadilla}{Pittsburgh}{Rochester}
%{Antibes}{Montpellier}{JuanLesPins}{Berkeley}
\usecolortheme{whale}
%{whale} {orchid}{owl}{albatross}{beaver}
\usefonttheme{professionalfonts}
%{professionalfonts} {serif}{structurebold}
%{structureitalicserif}{structuresmallcapsserif}
\useoutertheme{infolines}
%{infolines} {miniframes}{shadow}{sidebar}
%{smoothbars}{smoothtree}{split}{tree}
\useinnertheme{rectangles}
%{circles}{rounded}{rectangles}{inmargin}
\setbeamercovered{transparent=50}
}
\AtBeginSection[]
{
\begin{frame}
<beamer>
\frametitle{\sc plan de l'exposé}
\tableofcontents[currentsection,currentsubsection]

\end{frame}
}

\begin{document}
	\begin{frame}
		\titlepage
	\end{frame}
	\section{Motivation}
	\subsection{Rêves}
	\begin{frame}
	{Construction de la rêve dans la vie réelle}
	{Situation matrimoniale}
	\begin{itemize}
	\item Utilisez \texttt{itemize} à volonté.
	\item Utilisez des phrases très courtes et de 			courtes expressions.
	\end{itemize}
	\end{frame}
	\subsection{Vision du futur}
	\begin{frame}
	{Objectifs}
	{Réalisation}
	\lipsum[5]
	\end{frame}
	\section{Développement personnel}
	\subsection{Mentalité}
	\begin{frame}
	{Personnalité}
	{Confiance en soi}
	\begin{itemize}
	\item Utilisez \texttt{itemize} à volonté.
	\item Utilisez des phrases très courtes et de 			courtes expressions.
	\end{itemize}
	\end{frame}
	\subsection{Reveiller le meilleur en soi}
	\begin{frame}
	{méditation}
	{Psychologie}
	\lipsum[7]
	\end{frame}
	\begin{frame}
	{Faites des titres qui informent}
	Vous pouvez créer des recouvrements \dots
	\begin{itemize}
	\item en utilisant la commande \texttt{pause}:
	\begin{itemize}
	\item
	Premier item. \pause
	\item 
	Second item. \pause
	\end{itemize}
	\item en  utilisant les spécifications des recouvrements 
	\end{itemize}
	\end{frame}
\end{document}



